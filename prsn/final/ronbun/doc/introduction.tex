\newpage
\changeindent{0cm}
\section{はじめに}
\changeindent{2cm}
近年,機械学習の発展に伴い,様々な分野への応用がされており
様々な新規データセットにおいて目覚ましい結果が報告されている.
また新規データセットを生成する際,分類問題では各データにふさわしいラベル付けをする必要があるが,
ラベル付けには人の手が必要でありコストがかかる問題がある.
そこで半教師あり学習 (Semi-Supervised Learning: SSL)と呼ばれる
学習データ全体のうち一部にのみラベルが付与された状態で学習を行う手法が提案されており,
盛んに研究されている.また最新の SSL の手法では,全データにラベルが付与されている教師あり学習にも劣らない成果の報告\cite{sohn2020fixmatch}もある.しかし,当然のことながら各ラベルに対するデータ数が少ないほど精度が安定しなくなるという報告もされている.

一方,ラベルなしデータへのラベル付けタスクは一種の組合せ最適化問題と考えることができる.

本研究では,組合せ最適化問題に有効である遺伝的アルゴリズムを用いてデータにラベルを付与することでラベル付きデータが少ない場合における半教師あり学習の頑健
性を高める手法を提案する.

以下に本論文の構成を示す.まず,2章では本研究で用いる要素技術につ
いて概説する.続いて3章で実験手法の提案をし,4章において実験結果と考察を
示す.5章で本研究の成果をまとめたうえで,今後の課題について述べる.

