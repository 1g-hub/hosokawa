
\newpage
\changeindent{0cm}
\section{提案手法}
\changeindent{2cm}
本研究では,ラベルなしデータに対する疑似ラベルを GA を用いて
探索する手法を提案する.
また以降,半教師あり学習のデータについて学習データである
ラベル付きデータとラベルなしデータをそれぞれ $D_{\rm l}$\ ,$D_{\rm ul}$\ ,
テストデータを $D_{\rm t}$ と呼ぶ.


\changeindent{0cm}
\subsection{個体設定}
\changeindent{2cm}
まず,$D_{\rm ul}$からランダムにデータを取り出し探索データとし,各遺伝子座に一対一対応させる.
各遺伝子は探索データがとりうるラベルである整数値を持ち,
各遺伝子座に対応する探索データの疑似ラベルとする.
また,このとき探索データ数と遺伝子数が等しくなる.
以降探索データについて $D_{\rm s}$ と呼ぶ.

\changeindent{0cm}
\section{提案手法}
\changeindent{2cm}
以下に提案手法の手順を示す.

\begin{enumerate}
	\item モデルの事前学習
	\item 事前学習したモデルに入力データを $D_{\rm s}$ ,ラベルデータを各個体とし
	学習し,$D_{\rm l}$ の識別率を適応度として GA の探索を行う.
	\item ラベル付きデータとして $D_{\rm l}$ に探索された個体をラベルとして持つ $D_{s}$ を混ぜ
	再学習を行い $D_{\rm t}$ を識別する.
\end{enumerate}

既存の疑似ラベルを用いた手法では $D_{\rm s}$ で学習されるモデルの精度に大きく影響されるが,提案手法では疑似ラベルはモデルの出力ではないため正答率があがり,結果として半教師あり学習による精度も改善されることが期待できる.

