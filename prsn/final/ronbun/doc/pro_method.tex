
\newpage
\changeindent{0cm}
\section{提案手法}
\changeindent{2cm}
本研究では,ラベルなしデータに対する疑似ラベルを GA を用いて
探索する手法を提案する.
また以降,半教師あり学習のデータについて学習データである
ラベル付きデータとラベルなしデータをそれぞれ $D_{\rm l}$\ ,$D_{\rm ul}$\ ,
テストデータを $D_{\rm t}$ と呼ぶ.


\changeindent{0cm}
\subsection{個体設定}
\changeindent{2cm}
まず,$D_{\rm ul}$からランダムにいくつかデータを取り出し探索データとする.
ここで,GAの扱う個体は探索データに対する疑似ラベル群である.
探索データと各遺伝子座は一対一対応しており,
遺伝子型は対応するラベルを表す整数値である.
従って,遺伝子長は探索データ数となる.
以降探索データについて $D_{\rm s}$ と呼ぶ.

\changeindent{0cm}
\section{提案手法}
\changeindent{2cm}
以下に提案手法の手順を示す.

\begin{enumerate}
	\item モデルの事前学習
	\item 事前学習したモデルに入力データを $D_{\rm s}$ ,ラベルデータを各個体とし
	学習し,$D_{\rm l}$ の識別率を適応度として GA の探索を行う.
	\item $D_{\rm l}$ と探索された個体をラベルとして持つ $D_{s}$ とを合わせてラベル付きデータとして
	 SSL によって再学習を行い $D_{\rm t}$ を識別する.
\end{enumerate}

既存の疑似ラベルを用いた手法では $D_{\rm s}$ で学習されるモデルの精度に大きく影響されるが,提案手法では疑似ラベルはモデルの出力ではないため正答率があがり,結果として半教師あり学習による精度も改善されることが期待できる.

