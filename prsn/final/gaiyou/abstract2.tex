% \documentclass[a4paper,twoside,twocolumn,12pt]{article}
\documentclass[a4paper,twoside,twocolumn,10pt]{jarticle}     %pLaTeX2e仕様(platex.exeの場合)
\usepackage{abstract} % Style for abstracts in Dept. CSIS, OPU
%\usepackage{abstract4past} % Style for abstracts for the past curriculum

%%%%%%%%%% Designate packages you need %%%%%%%%%%
% \usepackage{graphicx} % Enhanced support for graphics
\usepackage[dvipdfm]{graphicx}
\usepackage{url} % Verbatim with URL-sensitive line breaks

%%%%%%%%%% Parameters that should be customized %%%%%%%%%%
% Language (1 = Japanese, 2 = English)
\setlang{1}
% Bachelor or Master (1 = Bachelor, 2 = Master)
\setborm{1}
% Fiscal year
\setfy{2021}
% Group number
\setgnum{1}
% Presentation order
\setorder{6}
% Increase page number (optional)
%% \pplus{1}

\title{遺伝的アルゴリズムによる機械学習における\\疑似ラベル生成手法の提案}
\author{細川 岳大}
%\ordering{6}[B]  %?²?_?ÍB?C?C?_?ÍM
\begin{document}
\maketitle
\small
\section{はじめに}
近年,コストなどの観点から分類問題において半教師あり学習という
学習データのうち一部にのみラベル付けされた状態で学習を行う手法が
盛んに研究されており,教師あり学習に近い精度を示す報告もされている.
一方で,各ラベルに対するデータが少ないほど精度にばらつきが出てしまう.
また,ラベル付けタスクについて一種の組み合わせ最適化問題ととらえることができる.

そこで本研究では,組み合わせ最適化問題の有効なアルゴリズムとして遺伝的アルゴリズムを用いて
ラベル付きデータを増やすことでラベル付きデータが非常に少ない場合における半教師あり学習の頑健
性を高めることを目的とする.
\section{要素技術}
\subsection{Contrastive Learning}
Contrastive Learning (CL) とは, 特徴表現を獲得するための
自己教師あり学習のひとつである.ある画像変換をした画像のペアについて
元画像が一致するか否かを識別するタスクであり,
一つの画像から得られる特徴表現が画像変換によって
画像の持つ意味があまり変化しないようなものを獲得することができる.
\subsection{SimCLR}
Simple framework for Contrastive Learning of visual Representation(SimCLR)\cite{chen2020simple} は半教師あり学習の手法の一つであり, CL により特徴抽出器を学習したうえで,少量のラベル付きデータによる Fintuning によって分類器を獲得する手法である.
\subsection{疑似ラベル}
疑似ラベル (Pseudo Label) はあるモデルによって予測されるラベルなしデータに対する暫定的なラベルである.
疑似ラベルを付与したデータをラベル付きデータに混ぜて学習することで正則化の役割をする.
\subsection{遺伝的アルゴリズム}
遺伝的アルゴリズム (Genetic Algorithm :GA) は生物の進化を模倣した最適解を探索するアルゴリズムである.
選択,交叉,突然変異の3つの操作によって解に対する多様性を保ちつつ最適解を探索することができる.
\section{提案手法}
本研究では, CIFAR-10 における一部のみをラベル付きデータとし,残りのラベルなしデータのうち一部を取り出し,
それに対するラベルを GA によって探索する.
GA の個体について各遺伝子はラベルデータのいずれかの整数値とし,取り出されたラベルなしデータに対する疑似ラベルとする.
以下に提案手法の概要を示す.
\begin{enumerate}
	\item CL よって特徴抽出器を学習する.
	\item GA の個体と対応するデータを学習データとし, Finetuning し,ラベル付きデータに対する識別率
	を適応度として GA の探索を行う.
	\item 探索された個体とラベル付きデータを学習データとし, Finetuning を行い,
	テストデータを識別する.
\end{enumerate}
\section{数値実験}
\subsection{実験方法}
データセットとして10クラス識別問題である CIFAR-10 を用いた.
学習に50000件,テストに10000件をそれぞれ使用し,ラベル付きデータ及び疑似ラベルをつける対象である
ラベルなしデータはどちらも学習データから選んだ.

CL の学習について,特徴抽出器には Resnet-18 を用い,学習データ50000件を, 500 epoch 学習した.
また,特徴抽出器の出力次元は 512 次元であり,分類器は MLP を使用した.
GA の適応度計算については,30 epoch 学習した.
\subsection{結果と考察}
表\ref{tb:ex1}に結果の一例を示す. 提案手法が最も低い精度となった.

一方で, GA の世代ごとの適応度のばらつきが小さくなっていることは確認できており探索については進んでいるといえ,
疑似ラベルに対する真の正答率は 30\% まで上がっていることが確認できた.

考察として GA の探索において今回の設定では各遺伝子が10個の離散値を持つために多様性を保つ必要性があり,
世代ごとの個体数などのパラメータの調整不足である点があげられ,そのため局所解に陥ってしまったのではないかと考えられる.また,ラベルの多くが間違いラベルのため train\_accuracy が学習において 80\% 以上にならず各個体に対しての正確な適応度として機能していないかのうせいもあげられる.
\
\begin{table}[h]
	\centering
	\caption{実験結果\label{tb:ex1}}
	\scalebox{1.0}{
		\begin{tabular}{|c|c|} \hline
			baseline (ラベル付きデータのみ)&0.772\\ \hline\hline
			生成ラベル&テスト識別率\\ \hline\hline
			正解ラベル&0.822\\ \hline
			baseline モデルによる疑似ラベル&0.784\\ \hline
			提案手法による疑似ラベル&0.674\\ \hline
		\end{tabular}
	}
\end{table}

\section{まとめと今後の課題}
本研究では, GA によるラベルなしデータに対する疑似ラベルの生成手法の提案した.

今後の課題として, GA の拡張手法や適応度の計算方法を工夫することでより
良質な疑似ラベルを生成し識別利向上を目指すことが挙げられる.

\bibliographystyle{unsrt}
\bibliography{sa}

\end{document}

